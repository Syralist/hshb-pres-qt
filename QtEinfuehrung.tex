\documentclass{beamer}

\usepackage[ngerman]{babel}
\usepackage{selinput}
\SelectInputMappings{%
   adieresis={ä},
   germandbls={ß},
   Euro={€}
   }

\usetheme{Hannover}  %% Themenwahl

\setbeamercovered{transparent}
%\setbeamertemplate{footline}[frame number]
 
\title{Qt - Eine Einführung}
\author{Thomas Helmke}
\date{25.06.2013}
 
\begin{document}
\maketitle
\frame{\tableofcontents}
 
\section{Was ist Qt?}
\begin{frame}[<+->] %%Eine Folie
	\frametitle{Was ist Qt?} %%Folientitel
	\begin{itemize}
		\item Plattformübergreifendes Framework
		\item Geschrieben in C++
		\item Wrapper für Python, Java
	\end{itemize}
\end{frame}

\section{Wofür Qt?}
\begin{frame}[<+->] %%Eine Folie
	\frametitle{Wofür benutzt man Qt?} %%Folientitel
	\begin{itemize}
		\item Graphische Oberflächen
		\item Plattformübergreifende Programmierung
		\item ...
	\end{itemize}
\end{frame}
\begin{frame} %%Eine Folie
	\frametitle{Lizensierung} %%Folientitel
	\begin{itemize}
		\item<1-2> GPL 3.0
			\begin{itemize}
				\item<2> Open Source
				\item<2> kostenlos
				\item<2> Kein Know-How-Protect möglich
			\end{itemize}
		\item<3-4> LGPL 2.0
			\begin{itemize}
				\item<4> Closed Source mit dynamischer Einbindung der Bibliotheken
				\item<4> kostenlos
				\item<4> Know-How-Protect möglich
			\end{itemize}
		\item<5-6> Proprietäre Lizenz
			\begin{itemize}
				\item<6> Closed Source mit statischer Einbindung der Bibliotheken
				\item<6> kostenpflichtig
				\item<6> Know-How-Protect möglich
			\end{itemize}
	\end{itemize}
\end{frame}

\section{Qt Creator}
\begin{frame}[<+->]
	\frametitle{Qt Creator}
	\begin{itemize}
		\item Qt Entwicklungsumgebung
		\item WYSIWYG GUI Editor
		\item Integrierte Hilfe
	\end{itemize}
\end{frame}
\end{document}